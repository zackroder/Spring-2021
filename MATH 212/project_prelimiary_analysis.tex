% Options for packages loaded elsewhere
\PassOptionsToPackage{unicode}{hyperref}
\PassOptionsToPackage{hyphens}{url}
%
\documentclass[
]{article}
\usepackage{lmodern}
\usepackage{amsmath}
\usepackage{ifxetex,ifluatex}
\ifnum 0\ifxetex 1\fi\ifluatex 1\fi=0 % if pdftex
  \usepackage[T1]{fontenc}
  \usepackage[utf8]{inputenc}
  \usepackage{textcomp} % provide euro and other symbols
  \usepackage{amssymb}
\else % if luatex or xetex
  \usepackage{unicode-math}
  \defaultfontfeatures{Scale=MatchLowercase}
  \defaultfontfeatures[\rmfamily]{Ligatures=TeX,Scale=1}
\fi
% Use upquote if available, for straight quotes in verbatim environments
\IfFileExists{upquote.sty}{\usepackage{upquote}}{}
\IfFileExists{microtype.sty}{% use microtype if available
  \usepackage[]{microtype}
  \UseMicrotypeSet[protrusion]{basicmath} % disable protrusion for tt fonts
}{}
\makeatletter
\@ifundefined{KOMAClassName}{% if non-KOMA class
  \IfFileExists{parskip.sty}{%
    \usepackage{parskip}
  }{% else
    \setlength{\parindent}{0pt}
    \setlength{\parskip}{6pt plus 2pt minus 1pt}}
}{% if KOMA class
  \KOMAoptions{parskip=half}}
\makeatother
\usepackage{xcolor}
\IfFileExists{xurl.sty}{\usepackage{xurl}}{} % add URL line breaks if available
\IfFileExists{bookmark.sty}{\usepackage{bookmark}}{\usepackage{hyperref}}
\hypersetup{
  pdftitle={MATH 212 Project Preliminary Analysis},
  pdfauthor={Zack Roder},
  hidelinks,
  pdfcreator={LaTeX via pandoc}}
\urlstyle{same} % disable monospaced font for URLs
\usepackage[margin=1in]{geometry}
\usepackage{color}
\usepackage{fancyvrb}
\newcommand{\VerbBar}{|}
\newcommand{\VERB}{\Verb[commandchars=\\\{\}]}
\DefineVerbatimEnvironment{Highlighting}{Verbatim}{commandchars=\\\{\}}
% Add ',fontsize=\small' for more characters per line
\usepackage{framed}
\definecolor{shadecolor}{RGB}{248,248,248}
\newenvironment{Shaded}{\begin{snugshade}}{\end{snugshade}}
\newcommand{\AlertTok}[1]{\textcolor[rgb]{0.94,0.16,0.16}{#1}}
\newcommand{\AnnotationTok}[1]{\textcolor[rgb]{0.56,0.35,0.01}{\textbf{\textit{#1}}}}
\newcommand{\AttributeTok}[1]{\textcolor[rgb]{0.77,0.63,0.00}{#1}}
\newcommand{\BaseNTok}[1]{\textcolor[rgb]{0.00,0.00,0.81}{#1}}
\newcommand{\BuiltInTok}[1]{#1}
\newcommand{\CharTok}[1]{\textcolor[rgb]{0.31,0.60,0.02}{#1}}
\newcommand{\CommentTok}[1]{\textcolor[rgb]{0.56,0.35,0.01}{\textit{#1}}}
\newcommand{\CommentVarTok}[1]{\textcolor[rgb]{0.56,0.35,0.01}{\textbf{\textit{#1}}}}
\newcommand{\ConstantTok}[1]{\textcolor[rgb]{0.00,0.00,0.00}{#1}}
\newcommand{\ControlFlowTok}[1]{\textcolor[rgb]{0.13,0.29,0.53}{\textbf{#1}}}
\newcommand{\DataTypeTok}[1]{\textcolor[rgb]{0.13,0.29,0.53}{#1}}
\newcommand{\DecValTok}[1]{\textcolor[rgb]{0.00,0.00,0.81}{#1}}
\newcommand{\DocumentationTok}[1]{\textcolor[rgb]{0.56,0.35,0.01}{\textbf{\textit{#1}}}}
\newcommand{\ErrorTok}[1]{\textcolor[rgb]{0.64,0.00,0.00}{\textbf{#1}}}
\newcommand{\ExtensionTok}[1]{#1}
\newcommand{\FloatTok}[1]{\textcolor[rgb]{0.00,0.00,0.81}{#1}}
\newcommand{\FunctionTok}[1]{\textcolor[rgb]{0.00,0.00,0.00}{#1}}
\newcommand{\ImportTok}[1]{#1}
\newcommand{\InformationTok}[1]{\textcolor[rgb]{0.56,0.35,0.01}{\textbf{\textit{#1}}}}
\newcommand{\KeywordTok}[1]{\textcolor[rgb]{0.13,0.29,0.53}{\textbf{#1}}}
\newcommand{\NormalTok}[1]{#1}
\newcommand{\OperatorTok}[1]{\textcolor[rgb]{0.81,0.36,0.00}{\textbf{#1}}}
\newcommand{\OtherTok}[1]{\textcolor[rgb]{0.56,0.35,0.01}{#1}}
\newcommand{\PreprocessorTok}[1]{\textcolor[rgb]{0.56,0.35,0.01}{\textit{#1}}}
\newcommand{\RegionMarkerTok}[1]{#1}
\newcommand{\SpecialCharTok}[1]{\textcolor[rgb]{0.00,0.00,0.00}{#1}}
\newcommand{\SpecialStringTok}[1]{\textcolor[rgb]{0.31,0.60,0.02}{#1}}
\newcommand{\StringTok}[1]{\textcolor[rgb]{0.31,0.60,0.02}{#1}}
\newcommand{\VariableTok}[1]{\textcolor[rgb]{0.00,0.00,0.00}{#1}}
\newcommand{\VerbatimStringTok}[1]{\textcolor[rgb]{0.31,0.60,0.02}{#1}}
\newcommand{\WarningTok}[1]{\textcolor[rgb]{0.56,0.35,0.01}{\textbf{\textit{#1}}}}
\usepackage{graphicx}
\makeatletter
\def\maxwidth{\ifdim\Gin@nat@width>\linewidth\linewidth\else\Gin@nat@width\fi}
\def\maxheight{\ifdim\Gin@nat@height>\textheight\textheight\else\Gin@nat@height\fi}
\makeatother
% Scale images if necessary, so that they will not overflow the page
% margins by default, and it is still possible to overwrite the defaults
% using explicit options in \includegraphics[width, height, ...]{}
\setkeys{Gin}{width=\maxwidth,height=\maxheight,keepaspectratio}
% Set default figure placement to htbp
\makeatletter
\def\fps@figure{htbp}
\makeatother
\setlength{\emergencystretch}{3em} % prevent overfull lines
\providecommand{\tightlist}{%
  \setlength{\itemsep}{0pt}\setlength{\parskip}{0pt}}
\setcounter{secnumdepth}{-\maxdimen} % remove section numbering
\ifluatex
  \usepackage{selnolig}  % disable illegal ligatures
\fi

\title{MATH 212 Project Preliminary Analysis}
\author{Zack Roder}
\date{4/18/2021}

\begin{document}
\maketitle

\hypertarget{data-set}{%
\subsection{Data Set}\label{data-set}}

From \url{https://fred.stlouisfed.org/}, I retrieved a data set with
various important economic indicators over the period from Q2 2011
through Q1 2020 (immediately before the COVID-induced recession).

\begin{Shaded}
\begin{Highlighting}[]
\NormalTok{econ\_indicators}
\end{Highlighting}
\end{Shaded}

\begin{verbatim}
## # A tibble: 36 x 6
##    observation_date    UNRATE    GDP MEDIAN_WEELKY_WAGES  DJCA HOUST
##    <dttm>               <dbl>  <dbl>               <dbl> <dbl> <dbl>
##  1 2011-04-01 00:00:00    9.1 15496.                 754 4271.  1723
##  2 2011-07-01 00:00:00    9   15592.                 760 4008.  1858
##  3 2011-10-01 00:00:00    8.5 15796.                 760 4068.  2015
##  4 2012-01-01 00:00:00    8.2 16020.                 764 4377.  2122
##  5 2012-04-01 00:00:00    8.2 16152.                 772 4365.  2218
##  6 2012-07-01 00:00:00    7.8 16257.                 766 4435.  2341
##  7 2012-10-01 00:00:00    7.9 16359.                 771 4413.  2724
##  8 2013-01-01 00:00:00    7.5 16570.                 768 4792.  2860
##  9 2013-04-01 00:00:00    7.5 16638.                 777 5091.  2604
## 10 2013-07-01 00:00:00    7.2 16849.                 779 5181.  2647
## # ... with 26 more rows
\end{verbatim}

\begin{enumerate}
\def\labelenumi{\arabic{enumi}.}
\tightlist
\item
  There are 5 variables in my data set.
\end{enumerate}

\begin{itemize}
\tightlist
\item
  Unemployment rate (UNRATE) - From FRED: ``The unemployment rate
  represents the number of unemployed as a percentage of the labor
  force. Labor force data are restricted to people 16 years of age and
  older, who currently reside in 1 of the 50 states or the District of
  Columbia, who do not reside in institutions (e.g., penal and mental
  facilities, homes for the aged), and who are not on active duty in the
  Armed Forces. This rate is also defined as the U-3 measure of labor
  underutilization.'' Unit is percent (seasonally adjusted).
\item
  Gross domestic product (GDP) - From FRED: ``Gross domestic product
  (GDP), the featured measure of U.S. output, is the market value of the
  goods and services produced by labor and property located in the
  United States.'' Unit is billions of dollars, seasonally adjusted
  annual rate.
\item
  Employed full time: median usual weekly nominal earnings, 16 years and
  over (MEDIAN\_EARNINGS) - from FRED: ``Data measure usual weekly
  earnings of wage and salary workers. Wage and salary workers are
  workers who receive wages, salaries, commissions, tips, payment in
  kind, or piece rates. The group includes employees in both the private
  and public sectors but, for the purposes of the earnings series, it
  excludes all self-employed persons, both those with incorporated
  businesses and those with unincorporated businesses. Usual weekly
  earnings represent earnings before taxes and other deductions and
  include any overtime pay, commissions, or tips usually received (at
  the main job in the case of multiple jobholders). Prior to 1994,
  respondents were asked how much they usually earned per week. Since
  January 1994, respondents have been asked to identify the easiest way
  for them to report earnings (hourly, weekly, biweekly, twice monthly,
  monthly, annually, or other) and how much they usually earn in the
  reported time period. Earnings reported on a basis other than weekly
  are converted to a weekly equivalent. The term''usual" is determined
  by each respondent's own understanding of the term. If the respondent
  asks for a definition of ``usual,'' interviewers are instructed to
  define the term as more than half the weeks worked during the past 4
  or 5 months." Unit is dollars, seasonally adjusted.
\item
  Dow Jones Composite Average (DJCA) - From FRED: ``The Dow Jones
  Composite Average is combination of all three major Dow Jones Averages
  (Industrial, Utility, and Transportation). Since the Composite Average
  is made up of this select group of prominent stocks, Dow Jones refers
  to it as a blue chip microcosm of the US stock market.'' Unit is
  index, not seasonally adjusted.
\item
  Housing Starts (HOUST) - From FRED: ``As provided by the Census, start
  occurs when excavation begins for the footings or foundation of a
  building. All housing units in a multifamily building are defined as
  being started when this excavation begins. Beginning with data for
  September 1992, estimates of housing starts include units in
  structures being totally rebuilt on an existing foundation.''
\end{itemize}

\begin{enumerate}
\def\labelenumi{\arabic{enumi}.}
\setcounter{enumi}{1}
\item
  My response variable in median weekly wages, while my predictor
  variables are unemployment, GDP, DJCA, and housing starts.
\item
  My research question is: can we linearly predict median weekly wages
  given some combination of unemployment, GDP, DJCA, and housing starts?
  I think this is an interesting question because median weekly wages
  provides a stronger indication of the economic conditions for actual
  working people in America. GDP, while important, doesn't tell us
  precisely if people are thriving or not. Hence, I want to see how
  median wages can be predicted from these other important economic
  indicators.
\end{enumerate}

\hypertarget{correlation-analysis}{%
\subsection{Correlation Analysis}\label{correlation-analysis}}

\begin{enumerate}
\def\labelenumi{\arabic{enumi}.}
\tightlist
\item
  Create correlation plot.
\end{enumerate}

\begin{Shaded}
\begin{Highlighting}[]
\NormalTok{MEDIAN\_WEEKLY\_WAGES }\OtherTok{\textless{}{-}}\NormalTok{ econ\_indicators}\SpecialCharTok{$}\NormalTok{MEDIAN\_WEELKY\_WAGES}
\NormalTok{UNRATE }\OtherTok{\textless{}{-}}\NormalTok{ econ\_indicators}\SpecialCharTok{$}\NormalTok{UNRATE}
\NormalTok{GDP }\OtherTok{\textless{}{-}}\NormalTok{ econ\_indicators}\SpecialCharTok{$}\NormalTok{GDP}
\NormalTok{DJCA }\OtherTok{\textless{}{-}}\NormalTok{ econ\_indicators}\SpecialCharTok{$}\NormalTok{DJCA}
\NormalTok{HOUST }\OtherTok{\textless{}{-}}\NormalTok{ econ\_indicators}\SpecialCharTok{$}\NormalTok{HOUST}

\NormalTok{DF\_indicators }\OtherTok{\textless{}{-}} \FunctionTok{data.frame}\NormalTok{(MEDIAN\_WEEKLY\_WAGES, UNRATE, GDP, DJCA, HOUST)}
\NormalTok{r }\OtherTok{\textless{}{-}} \FunctionTok{cor}\NormalTok{(DF\_indicators)}

\FunctionTok{ggcorrplot}\NormalTok{(r, }\AttributeTok{title=}\StringTok{"Economic Indicators (data from FRED)"}\NormalTok{, }\AttributeTok{lab=}\ConstantTok{TRUE}\NormalTok{)}
\end{Highlighting}
\end{Shaded}

\includegraphics{project_prelimiary_analysis_files/figure-latex/unnamed-chunk-2-1.pdf}

\begin{enumerate}
\def\labelenumi{\arabic{enumi}.}
\setcounter{enumi}{1}
\item
  The matrix indicates very high correlation between
  MEDIAN\_WEEKLY\_WAGES and both GDP and DJCA (0.98 and 0.96). Either of
  these predictor variables seem to be good candidates for being the
  first to enter our stepwise model.
\item
  All of our predictor values are correlated with our response variable
  (MEDIAN\_WEEKLY\_WAGES). UNRATE is negatively correlated, while HOUST,
  GDP, and DJCA are all positively correlated with our response
  variable.
\item
  Given how closely related some of these indicators are, we,
  unsurprisingly, do see multicolinearity in the dataset. For example,
  we see that GDP and DJCA are highly correlated (0.99).
\end{enumerate}

\hypertarget{variable-selection}{%
\subsection{Variable Selection}\label{variable-selection}}

\begin{enumerate}
\def\labelenumi{\arabic{enumi}.}
\tightlist
\item
  Regress MEDIAN\_WEEKLY\_WAGES on each predictor variable and compare
  outputs. Use alpha\_to\_enter = 0.15 and alpha\_to\_remove = 0.15.
\end{enumerate}

\begin{Shaded}
\begin{Highlighting}[]
\NormalTok{model1 }\OtherTok{\textless{}{-}} \FunctionTok{lm}\NormalTok{(MEDIAN\_WEEKLY\_WAGES }\SpecialCharTok{\textasciitilde{}}\NormalTok{ DJCA, }\AttributeTok{data=}\NormalTok{DF\_indicators)}
\FunctionTok{summary}\NormalTok{(model1)}
\end{Highlighting}
\end{Shaded}

\begin{verbatim}
## 
## Call:
## lm(formula = MEDIAN_WEEKLY_WAGES ~ DJCA, data = DF_indicators)
## 
## Residuals:
##     Min      1Q  Median      3Q     Max 
## -29.148  -8.956   5.366   9.865  41.847 
## 
## Coefficients:
##              Estimate Std. Error t value Pr(>|t|)    
## (Intercept) 6.068e+02  1.083e+01   56.04   <2e-16 ***
## DJCA        3.433e-02  1.639e-03   20.95   <2e-16 ***
## ---
## Signif. codes:  0 '***' 0.001 '**' 0.01 '*' 0.05 '.' 0.1 ' ' 1
## 
## Residual standard error: 15.41 on 34 degrees of freedom
## Multiple R-squared:  0.9281, Adjusted R-squared:  0.926 
## F-statistic:   439 on 1 and 34 DF,  p-value: < 2.2e-16
\end{verbatim}

\begin{Shaded}
\begin{Highlighting}[]
\NormalTok{model2 }\OtherTok{\textless{}{-}} \FunctionTok{lm}\NormalTok{(MEDIAN\_WEEKLY\_WAGES }\SpecialCharTok{\textasciitilde{}}\NormalTok{ GDP, }\AttributeTok{data=}\NormalTok{DF\_indicators)}
\FunctionTok{summary}\NormalTok{(model2)}
\end{Highlighting}
\end{Shaded}

\begin{verbatim}
## 
## Call:
## lm(formula = MEDIAN_WEEKLY_WAGES ~ GDP, data = DF_indicators)
## 
## Residuals:
##     Min      1Q  Median      3Q     Max 
## -17.242  -5.412  -1.048   6.200  32.190 
## 
## Coefficients:
##              Estimate Std. Error t value Pr(>|t|)    
## (Intercept) 2.930e+02  1.731e+01   16.92   <2e-16 ***
## GDP         2.893e-02  9.327e-04   31.02   <2e-16 ***
## ---
## Signif. codes:  0 '***' 0.001 '**' 0.01 '*' 0.05 '.' 0.1 ' ' 1
## 
## Residual standard error: 10.62 on 34 degrees of freedom
## Multiple R-squared:  0.9659, Adjusted R-squared:  0.9649 
## F-statistic: 962.4 on 1 and 34 DF,  p-value: < 2.2e-16
\end{verbatim}

\begin{Shaded}
\begin{Highlighting}[]
\NormalTok{model3 }\OtherTok{\textless{}{-}} \FunctionTok{lm}\NormalTok{(MEDIAN\_WEEKLY\_WAGES }\SpecialCharTok{\textasciitilde{}}\NormalTok{ UNRATE, }\AttributeTok{data=}\NormalTok{DF\_indicators)}
\FunctionTok{summary}\NormalTok{(model3)}
\end{Highlighting}
\end{Shaded}

\begin{verbatim}
## 
## Call:
## lm(formula = MEDIAN_WEEKLY_WAGES ~ UNRATE, data = DF_indicators)
## 
## Residuals:
##     Min      1Q  Median      3Q     Max 
## -37.204 -19.608  -3.611  15.235  85.521 
## 
## Coefficients:
##             Estimate Std. Error t value Pr(>|t|)    
## (Intercept)  990.160     14.999   66.01  < 2e-16 ***
## UNRATE       -28.791      2.536  -11.36 4.09e-13 ***
## ---
## Signif. codes:  0 '***' 0.001 '**' 0.01 '*' 0.05 '.' 0.1 ' ' 1
## 
## Residual standard error: 26.25 on 34 degrees of freedom
## Multiple R-squared:  0.7913, Adjusted R-squared:  0.7852 
## F-statistic: 128.9 on 1 and 34 DF,  p-value: 4.087e-13
\end{verbatim}

\begin{Shaded}
\begin{Highlighting}[]
\NormalTok{model4 }\OtherTok{\textless{}{-}} \FunctionTok{lm}\NormalTok{(MEDIAN\_WEEKLY\_WAGES }\SpecialCharTok{\textasciitilde{}}\NormalTok{ HOUST, }\AttributeTok{data=}\NormalTok{DF\_indicators)}
\FunctionTok{summary}\NormalTok{(model4)}
\end{Highlighting}
\end{Shaded}

\begin{verbatim}
## 
## Call:
## lm(formula = MEDIAN_WEEKLY_WAGES ~ HOUST, data = DF_indicators)
## 
## Residuals:
##    Min     1Q Median     3Q    Max 
## -44.95 -21.14  -6.13  21.36  45.23 
## 
## Coefficients:
##              Estimate Std. Error t value Pr(>|t|)    
## (Intercept) 5.906e+02  2.185e+01   27.03  < 2e-16 ***
## HOUST       7.409e-02  6.695e-03   11.07 8.19e-13 ***
## ---
## Signif. codes:  0 '***' 0.001 '**' 0.01 '*' 0.05 '.' 0.1 ' ' 1
## 
## Residual standard error: 26.79 on 34 degrees of freedom
## Multiple R-squared:  0.7827, Adjusted R-squared:  0.7763 
## F-statistic: 122.5 on 1 and 34 DF,  p-value: 8.189e-13
\end{verbatim}

Since the p-value for each of the predictor variables' slopes is less
than our alpha\_to\_enter, each is a candidate to be entered into the
stepwise model. We choose the one with the lowest p-value. Both the
MEDIAN\_WEEKLY\_WAGES \textasciitilde{} GDP and MEDIAN\_WEEKLY\_WAGES
\textasciitilde{} DJCA have slopes with p-value \textless{} 2e-16, but
we notice that the t-value for the slope in the GDP model is higher,
indicating greater significance. Thus, we'll choose GDP as the first
variable for our model.

\begin{enumerate}
\def\labelenumi{\arabic{enumi}.}
\setcounter{enumi}{1}
\tightlist
\item
  Fit each of the two-predictor models that include GDP as a predictor.
\end{enumerate}

\begin{Shaded}
\begin{Highlighting}[]
\NormalTok{model1 }\OtherTok{\textless{}{-}} \FunctionTok{lm}\NormalTok{(MEDIAN\_WEEKLY\_WAGES }\SpecialCharTok{\textasciitilde{}}\NormalTok{ GDP }\SpecialCharTok{+}\NormalTok{ DJCA, }\AttributeTok{data=}\NormalTok{DF\_indicators)}
\FunctionTok{summary}\NormalTok{(model1)}
\end{Highlighting}
\end{Shaded}

\begin{verbatim}
## 
## Call:
## lm(formula = MEDIAN_WEEKLY_WAGES ~ GDP + DJCA, data = DF_indicators)
## 
## Residuals:
##      Min       1Q   Median       3Q      Max 
## -17.9655  -6.9021  -0.1618   6.3199  30.0933 
## 
## Coefficients:
##               Estimate Std. Error t value Pr(>|t|)    
## (Intercept) 201.530534  63.562538   3.171  0.00328 ** 
## GDP           0.037562   0.005851   6.419 2.82e-07 ***
## DJCA         -0.010574   0.007083  -1.493  0.14495    
## ---
## Signif. codes:  0 '***' 0.001 '**' 0.01 '*' 0.05 '.' 0.1 ' ' 1
## 
## Residual standard error: 10.43 on 33 degrees of freedom
## Multiple R-squared:  0.968,  Adjusted R-squared:  0.9661 
## F-statistic: 499.7 on 2 and 33 DF,  p-value: < 2.2e-16
\end{verbatim}

\begin{Shaded}
\begin{Highlighting}[]
\NormalTok{model2 }\OtherTok{\textless{}{-}} \FunctionTok{lm}\NormalTok{(MEDIAN\_WEEKLY\_WAGES }\SpecialCharTok{\textasciitilde{}}\NormalTok{ GDP }\SpecialCharTok{+}\NormalTok{ UNRATE, }\AttributeTok{data=}\NormalTok{DF\_indicators)}
\FunctionTok{summary}\NormalTok{(model2)}
\end{Highlighting}
\end{Shaded}

\begin{verbatim}
## 
## Call:
## lm(formula = MEDIAN_WEEKLY_WAGES ~ GDP + UNRATE, data = DF_indicators)
## 
## Residuals:
##     Min      1Q  Median      3Q     Max 
## -12.865  -3.693  -1.033   3.978  17.201 
## 
## Coefficients:
##             Estimate Std. Error t value Pr(>|t|)    
## (Intercept) -5.52624   49.89618  -0.111    0.912    
## GDP          0.04086    0.00204  20.026  < 2e-16 ***
## UNRATE      13.82381    2.24317   6.163 5.98e-07 ***
## ---
## Signif. codes:  0 '***' 0.001 '**' 0.01 '*' 0.05 '.' 0.1 ' ' 1
## 
## Residual standard error: 7.348 on 33 degrees of freedom
## Multiple R-squared:  0.9841, Adjusted R-squared:  0.9832 
## F-statistic:  1024 on 2 and 33 DF,  p-value: < 2.2e-16
\end{verbatim}

\begin{Shaded}
\begin{Highlighting}[]
\NormalTok{model3 }\OtherTok{\textless{}{-}} \FunctionTok{lm}\NormalTok{(MEDIAN\_WEEKLY\_WAGES }\SpecialCharTok{\textasciitilde{}}\NormalTok{ GDP }\SpecialCharTok{+}\NormalTok{ HOUST, }\AttributeTok{data=}\NormalTok{DF\_indicators)}
\FunctionTok{summary}\NormalTok{(model3)}
\end{Highlighting}
\end{Shaded}

\begin{verbatim}
## 
## Call:
## lm(formula = MEDIAN_WEEKLY_WAGES ~ GDP + HOUST, data = DF_indicators)
## 
## Residuals:
##     Min      1Q  Median      3Q     Max 
## -15.734  -6.178  -0.912   5.764  35.614 
## 
## Coefficients:
##               Estimate Std. Error t value Pr(>|t|)    
## (Intercept) 254.963578  24.932722  10.226 9.22e-12 ***
## GDP           0.033336   0.002336  14.270 1.13e-15 ***
## HOUST        -0.013550   0.006645  -2.039   0.0495 *  
## ---
## Signif. codes:  0 '***' 0.001 '**' 0.01 '*' 0.05 '.' 0.1 ' ' 1
## 
## Residual standard error: 10.16 on 33 degrees of freedom
## Multiple R-squared:  0.9697, Adjusted R-squared:  0.9679 
## F-statistic:   528 on 2 and 33 DF,  p-value: < 2.2e-16
\end{verbatim}

Each second predictor (DJCA, UNRATE, HOUST) has a p-value \textless{}
0.15, and is thus a candidate to enter the model. We choose to proceed
with the one with the lowest p-value, which is UNRATE. Note that our
p-value for GDP remains below 0.15, so we proceed with the
MEDIAN\_WEEKLY\_WAGES \textasciitilde{} GDP + UNRATE model.

\begin{enumerate}
\def\labelenumi{\arabic{enumi}.}
\setcounter{enumi}{2}
\tightlist
\item
  Fit each three-predictor model that includes GDP and UNRATE.
\end{enumerate}

\begin{Shaded}
\begin{Highlighting}[]
\NormalTok{model1 }\OtherTok{\textless{}{-}} \FunctionTok{lm}\NormalTok{(MEDIAN\_WEEKLY\_WAGES }\SpecialCharTok{\textasciitilde{}}\NormalTok{ GDP }\SpecialCharTok{+}\NormalTok{ UNRATE }\SpecialCharTok{+}\NormalTok{ DJCA, }\AttributeTok{data=}\NormalTok{DF\_indicators)}
\FunctionTok{summary}\NormalTok{(model1)}
\end{Highlighting}
\end{Shaded}

\begin{verbatim}
## 
## Call:
## lm(formula = MEDIAN_WEEKLY_WAGES ~ GDP + UNRATE + DJCA, data = DF_indicators)
## 
## Residuals:
##      Min       1Q   Median       3Q      Max 
## -13.6477  -3.3376  -0.0371   2.9161  17.6972 
## 
## Coefficients:
##               Estimate Std. Error t value Pr(>|t|)    
## (Intercept) -61.818877  61.692482  -1.002    0.324    
## GDP           0.046589   0.004310  10.810 3.26e-12 ***
## UNRATE       13.470870   2.214328   6.084 8.53e-07 ***
## DJCA         -0.007391   0.004926  -1.501    0.143    
## ---
## Signif. codes:  0 '***' 0.001 '**' 0.01 '*' 0.05 '.' 0.1 ' ' 1
## 
## Residual standard error: 7.212 on 32 degrees of freedom
## Multiple R-squared:  0.9852, Adjusted R-squared:  0.9838 
## F-statistic:   709 on 3 and 32 DF,  p-value: < 2.2e-16
\end{verbatim}

\begin{Shaded}
\begin{Highlighting}[]
\NormalTok{model2 }\OtherTok{\textless{}{-}} \FunctionTok{lm}\NormalTok{(MEDIAN\_WEEKLY\_WAGES }\SpecialCharTok{\textasciitilde{}}\NormalTok{ GDP }\SpecialCharTok{+}\NormalTok{ UNRATE }\SpecialCharTok{+}\NormalTok{ HOUST, }\AttributeTok{data=}\NormalTok{DF\_indicators)}
\FunctionTok{summary}\NormalTok{(model2)}
\end{Highlighting}
\end{Shaded}

\begin{verbatim}
## 
## Call:
## lm(formula = MEDIAN_WEEKLY_WAGES ~ GDP + UNRATE + HOUST, data = DF_indicators)
## 
## Residuals:
##      Min       1Q   Median       3Q      Max 
## -12.8922  -4.0787  -0.2048   4.0326  17.8333 
## 
## Coefficients:
##               Estimate Std. Error t value Pr(>|t|)    
## (Intercept) -20.155789  53.014829  -0.380    0.706    
## GDP           0.040389   0.002124  19.012  < 2e-16 ***
## UNRATE       15.147181   2.743078   5.522 4.35e-06 ***
## HOUST         0.004972   0.005880   0.846    0.404    
## ---
## Signif. codes:  0 '***' 0.001 '**' 0.01 '*' 0.05 '.' 0.1 ' ' 1
## 
## Residual standard error: 7.38 on 32 degrees of freedom
## Multiple R-squared:  0.9845, Adjusted R-squared:  0.983 
## F-statistic: 676.7 on 3 and 32 DF,  p-value: < 2.2e-16
\end{verbatim}

Since the p-value for HOUST is greater than 0.15, it cannot be added to
the model. We thus proceed with MEDIAN\_WEEKLY\_WAGES \textasciitilde{}
GDP + UNRATE + DJCA.

\begin{enumerate}
\def\labelenumi{\arabic{enumi}.}
\setcounter{enumi}{3}
\tightlist
\item
  Fit each four-predictor model that includes GDP, UNRATE, and DJCA.
\end{enumerate}

\begin{Shaded}
\begin{Highlighting}[]
\NormalTok{model }\OtherTok{\textless{}{-}} \FunctionTok{lm}\NormalTok{(MEDIAN\_WEEKLY\_WAGES }\SpecialCharTok{\textasciitilde{}}\NormalTok{ GDP }\SpecialCharTok{+}\NormalTok{ UNRATE }\SpecialCharTok{+}\NormalTok{ DJCA }\SpecialCharTok{+}\NormalTok{ HOUST, }\AttributeTok{data=}\NormalTok{DF\_indicators)}
\FunctionTok{summary}\NormalTok{(model)}
\end{Highlighting}
\end{Shaded}

\begin{verbatim}
## 
## Call:
## lm(formula = MEDIAN_WEEKLY_WAGES ~ GDP + UNRATE + DJCA + HOUST, 
##     data = DF_indicators)
## 
## Residuals:
##      Min       1Q   Median       3Q      Max 
## -12.3470  -4.1344  -0.1523   3.5106  18.1108 
## 
## Coefficients:
##               Estimate Std. Error t value Pr(>|t|)    
## (Intercept) -68.095184  63.190012  -1.078    0.290    
## GDP           0.045830   0.004533  10.110 2.47e-11 ***
## UNRATE       14.439271   2.758068   5.235 1.09e-05 ***
## DJCA         -0.006848   0.005058  -1.354    0.186    
## HOUST         0.003541   0.005900   0.600    0.553    
## ---
## Signif. codes:  0 '***' 0.001 '**' 0.01 '*' 0.05 '.' 0.1 ' ' 1
## 
## Residual standard error: 7.285 on 31 degrees of freedom
## Multiple R-squared:  0.9853, Adjusted R-squared:  0.9835 
## F-statistic: 521.2 on 4 and 31 DF,  p-value: < 2.2e-16
\end{verbatim}

Note that the p-value for HOUST is greater than our alpha\_to\_enter, so
we do not include it in the model. Thus, our final regression model is
as follows:

\begin{Shaded}
\begin{Highlighting}[]
\NormalTok{model\_final }\OtherTok{\textless{}{-}} \FunctionTok{lm}\NormalTok{(MEDIAN\_WEEKLY\_WAGES }\SpecialCharTok{\textasciitilde{}}\NormalTok{ GDP }\SpecialCharTok{+}\NormalTok{ UNRATE }\SpecialCharTok{+}\NormalTok{ DJCA, }\AttributeTok{data=}\NormalTok{DF\_indicators)}
\FunctionTok{summary}\NormalTok{(model\_final)}
\end{Highlighting}
\end{Shaded}

\begin{verbatim}
## 
## Call:
## lm(formula = MEDIAN_WEEKLY_WAGES ~ GDP + UNRATE + DJCA, data = DF_indicators)
## 
## Residuals:
##      Min       1Q   Median       3Q      Max 
## -13.6477  -3.3376  -0.0371   2.9161  17.6972 
## 
## Coefficients:
##               Estimate Std. Error t value Pr(>|t|)    
## (Intercept) -61.818877  61.692482  -1.002    0.324    
## GDP           0.046589   0.004310  10.810 3.26e-12 ***
## UNRATE       13.470870   2.214328   6.084 8.53e-07 ***
## DJCA         -0.007391   0.004926  -1.501    0.143    
## ---
## Signif. codes:  0 '***' 0.001 '**' 0.01 '*' 0.05 '.' 0.1 ' ' 1
## 
## Residual standard error: 7.212 on 32 degrees of freedom
## Multiple R-squared:  0.9852, Adjusted R-squared:  0.9838 
## F-statistic:   709 on 3 and 32 DF,  p-value: < 2.2e-16
\end{verbatim}

\hypertarget{regression-model-and-interpretations}{%
\subsection{Regression Model and
Interpretations}\label{regression-model-and-interpretations}}

\begin{enumerate}
\def\labelenumi{\arabic{enumi}.}
\item
  We have thus found our final regression model. In standard form:
  \begin{align*}
  \text{MEDIAN_WEEKLY_WAGES} = -61.819 + 0.0466\text{GDP} + 13.471\text{UNRATE} - 0.00739\text{DJCA}
  \end{align*}
\item
  Coefficient Interpretation.
\end{enumerate}

\begin{itemize}
\tightlist
\item
  When
\end{itemize}

\end{document}
